\pagestyle{hedra}
\label{hedra}


\LeftImage{breve.jpg}

\noindent{}\textit{Labirintos do fascismo} é uma grande síntese de pesquisas de João Bernardo acerca do fascismo. Segundo o autor, ``a crítica do fascismo requer uma crítica do capitalismo e uma autocrítica do movimento operário'', ou seja, busca-se entender outras virtualidades contidas nas lutas anticapitalistas e expor os meandros de uma profunda derrota estratégica dos trabalhadores, e no que ela ainda pode resultar. %A partir da resenha: https://passapalavra.info/2018/08/122323/

\begin{ficha}
Editora: Hedra\\
Título: Labirintos do fascismo\\
Autor: João Bernardo (Vol.\,1)\\ 
ISBN: 978-65-89705-56-7\\
Páginas: XXX\\
Formato: 13,3x21\,cm\\
Preço: R\$ XX,XX\\
\end{ficha}

\pagebreak

\LeftImage{breve.jpg}

\noindent{}\textit{Labirintos do fascismo} é uma grande síntese de pesquisas de João Bernardo acerca do fascismo. Segundo o autor, ``a crítica do fascismo requer uma crítica do capitalismo e uma autocrítica do movimento operário'', ou seja, busca-se entender outras virtualidades contidas nas lutas anticapitalistas e expor os meandros de uma profunda derrota estratégica dos trabalhadores, e no que ela ainda pode resultar. %A partir da resenha: https://passapalavra.info/2018/08/122323/

\begin{ficha}
Editora: Hedra\\
Título: Labirintos do fascismo\\
Autor: João Bernardo (Vol.\,2)\\ 
ISBN: 978-65-89705-78-9\\
Páginas: XXX\\
Formato: 13,3x21\,cm\\
Preço: R\$ XX,XX\\
\end{ficha}

\pagebreak

\LeftImage{breve.jpg}

\noindent{}\textit{Labirintos do fascismo} é uma grande síntese de pesquisas de João Bernardo acerca do fascismo. Segundo o autor, ``a crítica do fascismo requer uma crítica do capitalismo e uma autocrítica do movimento operário'', ou seja, busca-se entender outras virtualidades contidas nas lutas anticapitalistas e expor os meandros de uma profunda derrota estratégica dos trabalhadores, e no que ela ainda pode resultar. %A partir da resenha: https://passapalavra.info/2018/08/122323/

\begin{ficha}
Editora: Hedra\\
Título: Labirintos do fascismo (Vol.\,3)\\
Autor: João Bernardo\\ 
ISBN: 978-65-89705-79-6\\
Páginas: XXX\\
Formato: 13,3x21\,cm\\
Preço: R\$ XX,XX\\
\end{ficha}

\pagebreak

\LeftImage{breve.jpg}

\noindent{}\textit{Labirintos do fascismo} é uma grande síntese de pesquisas de João Bernardo acerca do fascismo. Segundo o autor, ``a crítica do fascismo requer uma crítica do capitalismo e uma autocrítica do movimento operário'', ou seja, busca-se entender outras virtualidades contidas nas lutas anticapitalistas e expor os meandros de uma profunda derrota estratégica dos trabalhadores, e no que ela ainda pode resultar. %A partir da resenha: https://passapalavra.info/2018/08/122323/

\begin{ficha}
Editora: Hedra\\
Título: Labirintos do fascismo (Vol.\,4)\\
Autor: João Bernardo\\ 
ISBN: 978-65-89705-80-2\\
Páginas: XXX\\
Formato: 13,3x21\,cm\\
Preço: R\$ XX,XX\\
\end{ficha}

\pagebreak

\LeftImage{breve.jpg}

\noindent{}\textit{Labirintos do fascismo} é uma grande síntese de pesquisas de João Bernardo acerca do fascismo. Segundo o autor, ``a crítica do fascismo requer uma crítica do capitalismo e uma autocrítica do movimento operário'', ou seja, busca-se entender outras virtualidades contidas nas lutas anticapitalistas e expor os meandros de uma profunda derrota estratégica dos trabalhadores, e no que ela ainda pode resultar. %A partir da resenha: https://passapalavra.info/2018/08/122323/

\begin{ficha}
Editora: Hedra\\
Título: Labirintos do fascismo (Vol.\,5)\\
Autor: João Bernardo\\ 
ISBN: 978-65-89705-81-9\\
Páginas: XXX\\
Formato: 13,3x21\,cm\\
Preço: R\$ XX,XX\\
\end{ficha}

\pagebreak

\LeftImage{breve.jpg}

\noindent{}\textit{Labirintos do fascismo} é uma grande síntese de pesquisas de João Bernardo acerca do fascismo. Segundo o autor, ``a crítica do fascismo requer uma crítica do capitalismo e uma autocrítica do movimento operário'', ou seja, busca-se entender outras virtualidades contidas nas lutas anticapitalistas e expor os meandros de uma profunda derrota estratégica dos trabalhadores, e no que ela ainda pode resultar. %A partir da resenha: https://passapalavra.info/2018/08/122323/

\begin{ficha}
Editora: Hedra\\
Título: Labirintos do fascismo (Vol.\,6)\\
Autor: João Bernardo\\ 
ISBN: 978-65-89705-82-6\\
Páginas: XXX\\
Formato: 13,3x21\,cm\\
Preço: R\$ XX,XX\\
\end{ficha}

\pagebreak

\RightImage{breve.jpg}

\noindent{}\lipsum[2]

\begin{ficha}
Editora: Veneta \& Hedra\\
Título: Como derrotar o fascismo nas eleições\\
Autor: Sérgio Amadeu\\ 
ISBN: XXXXXXXXXXXXXXXXX\\
Páginas: XXX\\
Formato: 13,3x21\,cm\\
Preço: R\$ XX,XX\\
\end{ficha}


\pagebreak

\LeftImage{breve.jpg}

\noindent{}O livro traz dez contos do grande expoente do conto moderno japonês, publicados entre 1915 e 1927. \textit{Rashômon} e \textit{Dentro do bosque}, retratam a cultura de Heian (atual Quioto), enquanto outros temas explorados por Akutagawa são os antigos costumes japoneses, a ética cristã e a loucura (\textit{Memorando Ryôsai Ogata}, \textit{Ogin}, \textit{O mártir}). Já \textit{Devoção à literatura popular} e \textit{Terra morta} têm como fundo a cultura de Edo (atual Tóquio). Por fim, dois contos de caráter autobiográfico, do final da vida de Akutagawa: \textit{Passagens do caderno de notas de Yasukichi} e \textit{A vida de um idiota}. Esta nova edição, com texto revisto pelas tradutoras, conta ainda com nova introdução e acréscimo de notas.

\begin{ficha}
Editora: Hedra\\
Título: Rashômon e outros contos\\
Autor: Ryūnosuke Akutagawa\\ 
ISBN: 978-65-89705-59-8\\
Páginas: XXX\\
Formato: 13,3x21\,cm\\
Preço: R\$ XX,XX\\
\end{ficha}

\pagebreak

\RightImage{breve.jpg}

\noindent{}\textit{Teogonia} é um poema de 1022 versos hexâmetros datílicos que descreve a origem e a genealogia dos deuses. São narradas as peripécias que constituem o surgimento do universo e de seus deuses primordiais. Esta edição bilíngue do clássico grego conta com introdução e notas explicativas.

\begin{ficha}
Editora: Hedra\\
Título: Teogonia\\
Autor: Hesíodo\\ 
ISBN: 978-65-89705-58-1\\
Páginas: XXX\\
Formato: 13,3x21\,cm\\
Preço: R\$ XX,XX\\
\end{ficha}

\pagebreak

\LeftImage{breve.jpg}

\noindent{}\textit{Trabalhos e dias} é um poema épico de 828 versos em que são contados alguns dos mitos gregos mais conhecidos até hoje, como o de Prometeu e o de Pandora. Este poema é voltado para a condição dos mortais, explicitando suas necessidades e limitações, com foco no trabalho agrícola baseado nas estações do ano. Edição bilíngue.

\begin{ficha}
Editora: Hedra\\
Título: Trabalhos e dias\\
Autor: Hesíodo\\ 
ISBN: 978-65-89705-57-4\\
Páginas: XXX\\
Formato: 13,3x21\,cm\\
Preço: R\$ XX,XX\\
\end{ficha}

\pagebreak

\RightImage{breve.jpg}

\noindent{}\textit{Manifesto comunista}, publicado em 1848, é um dos textos mais influentes do mundo. Expõe o programa da Liga dos Comunistas, que encomendou o texto, e, contando com uma análise da luta de classes, tanto a partir de uma perspectiva histórica, quanto contemporânea, trata do período em que se estabelecia o capitalismo e, consequentemente, a burguesia como classe dominante, na Europa do século \textsc{xix}.

O \textit{Manifesto}, ainda que tenha incorporado elementos de outros pensadores, constituiu as bases da teoria sobre as classes sociais no capitalismo e a luta de classes, fundamentando os princípios do marxismo. Ainda que a autoria do Manifesto seja historicamente atribuída a Marx e Engels, este último foi responsável somente pela elaboração de seus primeiros rascunhos e a redação foi realizada por Marx.

\begin{ficha}
Editora: Hedra\\
Título: Karl Marx e Friedrich Engels\\
Autor: Manifesto comunista\\ 
ISBN: 978-65-89705-64-2\\
Páginas: XXX\\
Formato: 13,3x21\,cm\\
Preço: R\$ XX,XX\\
\end{ficha}

\pagebreak

\LeftImage{breve.jpg}

\noindent{}\textit{Memórias do subsolo} é um pequeno romance publicado em 1864. Considerada uma obra precursora do existencialismo e da psicanálise, traz na primeira parte o monólogo de um homem amargurado e amargo, um homem subterrâneo, sem nome ou relações sociais, um empregado aposentado, em cuja própria existência não vê nenhum sentido, e que se dirige diretamente ao leitor. Tenta envolvê-lo, convencê-lo e comovê-lo com hipóteses sobre si mesmo e sua possível redenção, talvez via a ação, nem que seja fazer o mal --- para afinal concluir que o melhor é não fazer nada. Na segunda parte, numa espécie de fluxo de consciência (técnica narrativa que seria mais tarde levada ao limite por Joyce), surgem as duras lembranças de situações e discursos que, numa sociedade hierarquizada, submetem e emparedam os humilhados e ofendidos.


\begin{ficha}
Editora: Hedra\\
Título: Memórias do subsolo\\
Autor: Fiódor Dostoiévski\\ 
ISBN: 978-65-89705-60-4\\
Páginas: XXX\\
Formato: 13,3x21\,cm\\
Preço: R\$ XX,XX\\
\end{ficha}

\pagebreak

\RightImage{breve.jpg}

\noindent{}O texto que dá título a esta edição é um \textit{fragmento de romance} publicado em 1840 e o ensejo imediato para a narrativa, que pretendia ser um romance histórico, foi a a intenção de Heine de contrapor-se à escalada de antissemitismo que se manifestava então na Alemanha. \textit{Rabi de Bacherach} é situado no final do século \textsc{xv}, mas o narrador remonta também a séculos anteriores para tocar nas raízes históricas do antissemitismo na Alemanha. Apesar de seu entusiasmo pelo projeto, Heine não conseguiu vencer a amplitude e a aspereza do assunto. Entretanto, mesmo em seu caráter fragmentário, a obra constitui expressivo exemplo da arte narrativa de Heine e é o documento mais elucidativo de sua flutuante relação com o judaísmo. Os três textos publicados como apêndice enfocam a questão do fanatismo religioso e foram extraídos do volume Lutetia, em que Heine enfeixou 61 artigos escritos em Paris entre fevereiro de 1840 e maio de 1844, bem como quatro artigos dos anos 1843 a 1846.

\begin{ficha}
Editora: Hedra\\
Título: O Rabi de Bacherach e três artigos sobre o ódio racial\\
Autor: Heinrich Heine\\ 
ISBN: 978-65-89705-06-2\\
Páginas: XXX\\
Formato: 13,3x21\,cm\\
Preço: R\$ XX,XX\\
\end{ficha}

\pagebreak

\LeftImage{breve.jpg}

\noindent{}Os Hupd’äh têm muitas histórias sobre a \textit{gente-sombra}. Os homens e mulheres-sombra são muito perigosos e usam roupas coloridas --- além de caçar e fazer mal aos Hup. Uma dessas roupas tem cor de sombra, daí seu nome. A gente-sombra causa doenças e pode até matar. Eles comem a carne e o espírito dos humanos. Mas muitos deles são sábios e conhecem cantos, mitos e benzimentos. Os cantos do homem-sombra é a história do encontro de um Hup com um homem-sombra chamado Way Naku. Os Hupd’äh (que até hoje alfabetizam suas crianças em Hup; muito diferente do português por ser uma língua tonal) vivem em aldeias espalhadas pela floresta amazônica, na região do Alto Rio Negro, fronteira entre o Brasil e a Colômbia.

\begin{ficha}
Editora: Hedra\\
Título: Os cantos do homem-sombra\\
Autor: Patience Pepps e Danilo Paiva Ramos\\ 
ISBN: 978-65-89705-73-4\\
Páginas: XXX\\
Formato: 13,3x21\,cm\\
Preço: R\$ XX,XX\\
\end{ficha}

\pagebreak

\RightImage{breve.jpg}

\noindent{}\textit{Os comedores de terra} é iniciado com a seguinte frase: \textit{Essa é a história dos nossos antepassados que aos poucos se multiplicaram}. Ela começa na época em que não havia Yanomami como os de hoje. Os comedores de terra sofriam, porque eles comiam terra. Nós também quase teríamos sofrido, como as minhocas, por cavar a terra e tomar vinho de barro, se não fossem os acontecimentos que seguem.

\begin{ficha}
Editora: Hedra\\
Título: Os Comedores de terra\\
Autor:  Pajés Parahiteri\\ 
ISBN: 978-65-89705-68-0\\
Páginas: XXX\\
Formato: 13,3x21\,cm\\
Preço: R\$ XX,XX\\
\end{ficha}

\pagebreak

\LeftImage{breve.jpg}

\noindent{}\textit{O surgimento dos pássaros} é iniciado com a seguinte frase: Como se chamava aquele que dividiu a terra, quando, no início, os Yanomami ainda não praticavam esse ritual; que, em seguida, se tornou líder e deu a terra? Ele tem nome. Aquele que mostrou o \textit{kawaamouse} chamava Gavião.

\begin{ficha}
Editora: Hedra\\
Título: O Surgimento dos pássaros\\
Autor:  Pajés Parahiteri\\ 
ISBN: 978-65-89705-70-3\\
Páginas: XXX\\
Formato: 13,3x21\,cm\\
Preço: R\$ XX,XX\\
\end{ficha}

\pagebreak

\RightImage{breve.jpg}

\noindent{}\textit{O surgimento da noite} é iniciado com a seguinte frase: \textit{Esta é a verdadeira história de nosso surgimento: quando a floresta era virgem, apareceu Horonami, personagem principal de nossa história, por causa de seus ensinamentos}. O grande pajé yanomami Horonami surgiu dele mesmo; surgiu ao mesmo tempo que esta floresta e foi quem ensinou os Yanomami a morar nela.

\begin{ficha}
Editora: Hedra\\
Título: O Surgimento da noite\\
Autor:  Pajés Parahiteri\\ 
ISBN: 978-65-89705-71-0\\
Páginas: XXX\\
Formato: 13,3x21\,cm\\
Preço: R\$ XX,XX\\
\end{ficha}

\pagebreak

\LeftImage{breve.jpg}

\noindent{}\textit{A árvore dos cantos} é iniciado com a seguinte frase: \textit{Nós vamos cantar. No início, não havia canto, não havia, ninguém cantava. Onde se erguia a árvore dos cantos, os dois foram caçar. Dois moços Wakusitari --- dois não, um só moço, que a descobriu em sua região}.

\begin{ficha}
Editora: Hedra\\
Título: A Árvore dos cantos\\
Autor:  Pajés Parahiteri\\ 
ISBN: 978-65-89705-69-7\\
Páginas: XXX\\
Formato: 13,3x21\,cm\\
Preço: R\$ XX,XX\\
\end{ficha}

\pagebreak

\RightImage{breve.jpg}

\noindent{}\textit{A terra uma só (Yvy Rupa)}, de Timóteo da Silva Verá Tupã Popygua, liderança guarani, conta o que aprendeu e pensou nos caminhos que percorreu pela Mata Atlântica, na América do Sul, junto ao seu povo Guarani Mbya. Suas narrativas sobre a origem da terra, do ser humano, da linguagem humana e dos animais e plantas da Mata Atlântica foram documentadas e traduzidas pela primeira vez por León Cadogan em Ayvú Rapyta. 

Ao longo de mais de uma década, o autor Timóteo e a organizadora Anita Ekman leram e discutiram as traduções do Avvy Rapta de Cadogan. Inspirada pelos poéticos comentários de Timóteo, bem como sua necessidade de documentar e divulgar a luta pela terra Guarani e a preservação da Mata Atlântica (\textit{Ka’a porã}), Ekman incentivou Timóteo ao desafio de ser o primeiro Guarani Mbya a contar diretamente em português, sem intermediário de um jurua (não indígena), sua versão da história do mundo e do seu povo.

\begin{ficha}
Editora: Hedra\\
Título: A terra uma só\\
Autor:  Timóteo da Silva Verá Tupã Popyguá\\ 
ISBN: 978-65-89705-66-6\\
Páginas: XXX\\
Formato: 13,3x21\,cm\\
Preço: R\$ XX,XX\\
\end{ficha}

\pagebreak

\LeftImage{breve.jpg}

\noindent{}Os quase oito mil Caxinauá fazem parte da família linguística pano, composta por cerca de trinta grupos, ocupando a fronteira entre o Brasil e o Peru. No Brasil, eles vivem em doze terras indígenas e, no Peru, eles ocupam todo o rio Curanja e uma parte do rio Purus — da cidade de Puerto Esperanza até a embocadura do rio Curanja. O historiador João Capistrano de Abreu foi quem, no início do século \textsc{xx}, registrou pela primeira vez a língua e o modo de vida Caxinauá junto a dois jovens provenientes da etnia, do rio Ibuaçu. 

Esse trabalho deu origem ao livro Hantxa huni kuin (1914), sobre a língua dos Caxinauá do rio Ibuaçu, afluente do Muru. Hoje a língua não é mais escrita do modo que Capistrano a registrou, e os próprios Caxinauá não conseguem ler esses relatos de cem anos atrás. Para tornar a história deste povo mais acessível, a linguista Eliane Camargo, que trabalha com eles desde 1987, revisou e refez a tradução da publicação já ultrapassada de Capistrano.

\begin{ficha}
Editora: Hedra\\
Título: A Mulher que virou Tatu\\
Autor:  Eliane Camargo\\ 
ISBN: 978-65-89705-72-7\\
Páginas: XXX\\
Formato: 13,3x21\,cm\\
Preço: R\$ XX,XX\\
\end{ficha}

\pagebreak

\RightImage{breve.jpg}

\noindent{}\textit{Nas redes guarani} reúne textos de autores guarani, de antropólogos vinculados ao Centro de Estudos Ameríndios da Universidade de São Paulo (\textsc{cesta-usp}) e de pesquisadores convidados de outras instituições. A coletânea se volta para redes guarani de pessoas, lugares e práticas de conhecimentos, bem como se reconhece como parte delas.

\begin{ficha}
Editora: Hedra\\
Título: Nas redes guarani\\
Autor: Dominique Tilkin Gallois e Valéria Macedo\\ 
ISBN: 978-65-89705-75-8\\
Páginas: XXX\\
Formato: 13,3x21\,cm\\
Preço: R\$ XX,XX\\
\end{ficha}

\pagebreak

\LeftImage{breve.jpg}

\noindent{}\textit{Círculos de coca e fumaça} debruça-se sobre os Hupd’äh, povo indígena falante de língua Hup que vive na região do Alto Rio Negro, no noroeste da Amazônia. Suas rodas noturnas para ingerir coca e tabaco --- momentos de compartilhar mitos e histórias de andanças pela mata, ensinar benzimentos e executar curas e proteções xamânicas --- são o principal cenário do livro. 

Nessas situações, Paiva Ramos percebeu performances, contextos em que os ameríndios relacionam suas experiências e observações da mata com as palavras dos mitos e encantamentos. A partir dessa interação, o viajante hup consegue interagir com seres de múltiplas paisagens e expandir seu campo de percepção, em um engajamento mútuo com os processos de transformação do mundo.

\begin{ficha}
Editora: Hedra\\
Título: Círculos de coca e fumaça\\
Autor:  Danilo Paiva Ramos\\ 
ISBN: 978-65-89705-74-1\\
Páginas: XXX\\
Formato: 13,3x21\,cm\\
Preço: R\$ XX,XX\\
\end{ficha}

\pagebreak

\RightImage{breve.jpg}

\noindent{}\textit{Crônicas de caça e criação} é uma pesquisa etnográfica sobre os Awá-Guajá, povo de língua Tupi-Guarani da Amazônia Oriental, precisamente do noroeste do Maranhão. Um dos últimos povos indígenas a serem contatados pelo Estado brasileiro, é constituído por caçadores habilidosos, que estruturam grande parte de seu sistema de pensamento nessa atividade e passaram a viver em aldeias após o contato iniciado pela Funai. No livro, Uirá Garcia se debruça, principalmente, sobre as relações que os Guajá estabelecem com seu território, assim como suas concepções cartográficas; suas formas de pensar a pessoa humana; a construção dos parentescos; a caça como atividade central da vida; e a relação dos humanos com os karawara, entidades que habitam esferas celestiais. Para apreender e transmitir tal sistema de vida, o antropólogo passou treze meses entre os Guajá, momento em que frequentou as aldeias Juriti, Tiracambu e Awá.

\begin{ficha}
Editora: Hedra\\
Título: Crônicas de caça e criação\\
Autor:  Uirá Garcia\\ 
ISBN: 978-65-89705-76-5\\
Páginas: XXX\\
Formato: 13,3x21\,cm\\
Preço: R\$ XX,XX\\
\end{ficha}

\pagebreak

\LeftImage{breve.jpg}

\noindent{}\lipsum[1]

\begin{ficha}
Editora: Hedra\\
Título: Imbodnoko taso: Não havia mais homens\\
Autor:  Luciana Storto\\ 
ISBN: 978-65-89705-77-2\\
Páginas: XXX\\
Formato: 13,3x21\,cm\\
Preço: R\$ XX,XX\\
\end{ficha}

\pagebreak
\pagestyle{hedracat}

\begin{multicols}{2}
\begin{enumerate}
\raggedright\nohyphens{
\item Ecopolítica, {\Formular{\textbf{Edson Passetti (org.)}}}	
\item Mare nostrum: Paranã Tipi, {\Formular{\textbf{Fabio Atui}}}
\item Crônicas de caça e criação, {\Formular{\textbf{Uirá Garcia}}}
\item Nas redes guarani, {\Formular{\textbf{Valéria Macedo}}}
\item A constituição traída, {\Formular{\textbf{Cleonildo Cruz (org.)}}}
\item Diário de um escritor na Rússia, {\Formular{\textbf{Flávio Ricardo Vassoler}}}
\item Lugar de negro, lugar de branco?, {\Formular{\textbf{Douglas Rodrigues Barros}}}
\item A sociedade de controle, {\Formular{\textbf{Sergio Amadeu (org.)}}}
\item O renascimento do autor, {\Formular{\textbf{Caio Gagliardi}}}
\item O que eu vi o que nós veremos, {\Formular{\textbf{Santos Dumont}}}
\item O outro lado da moeda (Teleny), {\Formular{\textbf{Oscar Wilde}}}
\item Imagens de um mundo trêmulo, {\Formular{\textbf{John Milton}}}
\item Michel Temer e o fascismo comum, {\Formular{\textbf{Tales Ab'Sáber}}}
\item Ao longo do rio, {\Formular{\textbf{Alexandre Koji Shiguehara}}}
\item Solombra, ou a sombra que cai sobre o eu, {\Formular{\textbf{João Adolfo Hansen}}}
\item Joana d'Arc, {\Formular{\textbf{Jules Michelet}}}
\item O coletivo aleatório, {\Formular{\textbf{Luis Marra}}}
\item A história das religiões na cultura moderna, {\Formular{\textbf{Marcello Massenzio}}}
\item Cordel - F. das Chagas Batista, {\Formular{\textbf{Francisco das Chagas Batista}}}
\item Elixir do pajé, {\Formular{\textbf{Bernardo Guimarães}}}
\item Cordel - João Martins de Athayde, {\Formular{\textbf{João Martins de Athayde}}}
\item Modos de representação da Bienal de São Paulo, {\Formular{\textbf{Vinicius Spricigo}}}
\item Padeirinho da Mangueira: retrato sincopado de um artista, {\Formular{\textbf{Franco Paulino}}}
\item Do futuro e da morte do teatro brasileiro, {\Formular{\textbf{Christina Barros Riego}}}
\item Canudos, história em versos, {\Formular{\textbf{Manuel Pedro das Dores Bombinho}}}
\item O cego e outros contos, {\Formular{\textbf{D. H. Lawrence}}}
\item Poesia seiscentista
\item Monoteísmos e dualismos: as religiões da salvação, {\Formular{\textbf{Giovanni Filoramo}}}
\item Apologia de Galileu, {\Formular{\textbf{Tommaso Campanella}}}
\item Flor do deserto, {\Formular{\textbf{Waris Dirie; Cathleen Miller}}}
\item Cinco lugares da fúria, {\Formular{\textbf{Pádua Fernandes}}}
\item O livro dos mandamentos, {\Formular{\textbf{Maimônides}}}
\item A conjuração de Catilina, {\Formular{\textbf{Salústio}}}
\item Fábula de Polifemo e Galatéia e outros poemas, {\Formular{\textbf{Góngora}}}
\item Histórias de igrejas destruídas, {\Formular{\textbf{Eduardo Brigagão Verderame}}}
\item Performances, {\Formular{\textbf{Brian Friel}}}
\item Cultura pop japonesa, {\Formular{\textbf{Sonia Bide Luyten}}}
\item História trágica do doutor Fausto, {\Formular{\textbf{Christopher Marlowe}}}
\item Micromegas, {\Formular{\textbf{Voltaire}}}
\item Politeísmos: as religiões do mundo antigo, {\Formular{\textbf{Paolo Scarpi}}}
\item Triunfos, {\Formular{\textbf{Petrarca}}}
\item Museu arte hoje, {\Formular{\textbf{Martin Grossmann; Gilberto Mariotti}}}
\item Viagem sentimental, {\Formular{\textbf{Laurence Sterne}}}
\item A Arte de olhar diferente, {\Formular{\textbf{Braulio Tavares}}}
\item O Pequeno Zacarias chamado Cinábrio, {\Formular{\textbf{E.T.A. Hoffman}}}
\item Oliver Twist (Bolso), {\Formular{\textbf{Charles Dickens}}}
\item Alegoria - Construção e interpretação da metáfora, {\Formular{\textbf{João Adolfo Hansen}}}
\item Teatro do êxtase, {\Formular{\textbf{Fernando Pessoa}}}
\item Paulo Whitaker, {\Formular{\textbf{Paulo Whitaker}}}
\item Todas as coisas pequenas, {\Formular{\textbf{Noemi Jaffe}}}
\item Questão do fim da arte em Hegel, {\Formular{\textbf{Marco Aurélio Werle}}}
\item Tratados da terra e gente do Brasil, {\Formular{\textbf{Fernão Cardim}}}
\item Dos nervos, {\Formular{\textbf{Ricardo Lísias}}}
\item Adeus ponta do meu nariz, {\Formular{\textbf{Edward Lear}}}
\item Cidade ampliada, {\Formular{\textbf{Rodrigo José Fermino}}}
\item O diário perdido do Jardim Maia, {\Formular{\textbf{Luís Marra}}}
\item Sobre a filosofia e outros diálogos, {\Formular{\textbf{Jorge Luis Borges; Osvaldo Ferrari}}}
\item Cordel: Franklin Maxado, {\Formular{\textbf{Franklin Maxado}}}
\item Dos novos sistemas na arte, {\Formular{\textbf{Kazimir Malievitch}}}
\item Cordel: Cuíca de Santo Amaro, {\Formular{\textbf{Cuíca de Santo Amaro}}}
\item Manual da destruição, {\Formular{\textbf{Alexandre Dal Farra}}}
\item A imprensa carnavalesca no Brasil, {\Formular{\textbf{José Ramos Tinhorão}}}
\item Índia e Extremo Oriente: via da libertação e da imortalidade, {\Formular{\textbf{Massimo Raveri}}}
\item Leitores e leituras de Clarice Lispector
\item Círculos de coca e fumaça, {\Formular{\textbf{Danilo Paiva Ramos}}}
\item Cordel: Severino José, {\Formular{\textbf{Severino José}}}
\item Escritório; O Espaço da Produção, {\Formular{\textbf{Claudio Silveira Amaral}}}
\item As minas de Salomão, {\Formular{\textbf{Rider Haggard}}}
\item Crédito à morte, {\Formular{\textbf{Anselm Jappe}}}
\item A cidade e as serras, {\Formular{\textbf{Eça de Queiroz}}}
\item Oliver Twist, {\Formular{\textbf{Charles Dickens}}}
\item Dao De Jing, {\Formular{\textbf{Lao Zi}}}
\item Sobre a amizade e outros diálogos, {\Formular{\textbf{Jorge Luis Borges; Osvaldo Ferrari}}}
\item Aqui tem coisa, {\Formular{\textbf{Patativa do Assaré}}}
\item Dicionário livre do santome-português, {\Formular{\textbf{Araújo \& Hagemeijer}}}
\item Aqui tem coisa, {\Formular{\textbf{Patativa do Assaré}}}
\item Imagem contemporânea I
\item Cordel - J. Borges, {\Formular{\textbf{José Francisco Borges}}}
\item Exato acidente, {\Formular{\textbf{Tony Monti}}}
\item Woyzeck, {\Formular{\textbf{George Buchner}}}
\item Autobiografia de um super-herói, {\Formular{\textbf{Alexandre Barbosa de Souza}}}
\item O menino da rosa, {\Formular{\textbf{Tony Monti}}}
\item Cordel - Rouxinol do Rinaré, {\Formular{\textbf{Rouxinol do Rinaré}}}
\item Imagem contemporânea II
\item História da província Santa Cruz, {\Formular{\textbf{Pero de Magalhães Gandavo}}}
\item Édipo Rei, {\Formular{\textbf{Sófocles}}}
\item Cordel - José Soares, {\Formular{\textbf{José Soares}}}
\item Greve contra a guerra, {\Formular{\textbf{Ricardo Lísias}}}
\item Cidade anônima, {\Formular{\textbf{Beatriz Furtado}}}
\item Primeiro de abril, {\Formular{\textbf{André Luiz Pinto}}}
\item Cordel: Oliveira de Panelas, {\Formular{\textbf{Oliveira de Panelas}}}
\item Fazendo Rizoma
\item Uma história do futebol, {\Formular{\textbf{Bill Murray}}}
\item Gangorra, {\Formular{\textbf{Regina Sawaya}}}
\item Poesia vaginal, {\Formular{\textbf{Glauco Mattoso}}}
\item Cultura popular - uma introdução, {\Formular{\textbf{Dominic Strinati}}}
\item Vocabulário de música pop, {\Formular{\textbf{Roy Shuker}}}
\item A invenção da pornografia, {\Formular{\textbf{Lynn Hunt}}}
\item Eu conheci Benny Moré
\item Deriva, {\Formular{\textbf{André Fernandes}}}
\item Fedro, {\Formular{\textbf{Platão}}}
\item Sobre os sonhos e outros diálogos, {\Formular{\textbf{Jorge Luis Borges; Osvaldo Ferrari}}}
\item O sapo voador, {\Formular{\textbf{Ademir Barbosa Jr.}}}
\item Arcana coelestia e Apocalipsis revelata, {\Formular{\textbf{Emanuel Swedenborg}}}
\item Letra de forma, {\Formular{\textbf{Laura Estelita Teixeira}}}
\item Os cães de que desistimos, {\Formular{\textbf{Chantal Castel}}}
\item Cordel - Téo Azevedo, {\Formular{\textbf{Téo Azevedo}}}
\item O que eu vi, o que nós veremos [bolso], {\Formular{\textbf{Santos Dumont}}}
\item A Fábrica de robôs, {\Formular{\textbf{Karel Tchápek}}}
\item Folhas de relva, {\Formular{\textbf{Walt Whitman}}}
\item Helio Piñon : Ideias e formas, {\Formular{\textbf{Pfeiffe, Helen; Ana Rosa}}}
\item O Rabi de Bacherach e três artigos sobre o ódio racial, {\Formular{\textbf{Heinrich Heine}}}
\item Refugiados de Idomeni, {\Formular{\textbf{Gabriel Bonis}}}
\item Visão de psicanálise, {\Formular{\textbf{Renato Bulcão}}}
\item Viagem em volta do meu quarto, {\Formular{\textbf{Xavier de Maistre}}}
\item Contos clássicos de vampiro, {\Formular{\textbf{Lord Byron; Bram Stoker}}}
\item Cultura estética e liberdade, {\Formular{\textbf{Friedrich Von Schiller}}}
\item Dostoiévski e a dialética, {\Formular{\textbf{Flávio Ricardo Vassoler}}}
\item Cabeças e outros poemas, {\Formular{\textbf{Pedro Luis Marques de Armas}}}
\item Razão sangrenta, {\Formular{\textbf{Robert Kurz}}}
\item A Velha Izerguil e outros contos, {\Formular{\textbf{Maksim Górki}}}
\item Viagem à turquia, bálcãs e Egito, {\Formular{\textbf{John Milton}}}
\item Do sentimento trágico da vida, {\Formular{\textbf{Miguel de Unamuno}}}
\item Rashômon e outros contos, {\Formular{\textbf{Akutagawa}}}
\item Feitiço de amor e outros contos, {\Formular{\textbf{Johann Ludwig Tieck}}}
\item Ode ao Vento Oeste e outros poemas, {\Formular{\textbf{P. B. Shelley}}}
\item Esperança do mundo, {\Formular{\textbf{Albert Camus}}}
\item Universidade, cidade, cidadania, {\Formular{\textbf{Franklin Leopoldo e Silva}}}
\item Estado crítico, {\Formular{\textbf{Régis Bonvicino}}}
\item Poemas da cabana montanhesa, {\Formular{\textbf{Saigyo}}}
\item Dançando em Lúnassa, {\Formular{\textbf{Brian Frield}}}
\item Lulismo, carisma pop e cultura anticrítica, {\Formular{\textbf{Tales Ab'Sáber}}}
\item Utopia Brasil, {\Formular{\textbf{Darcy Ribeiro}}}
\item Americanismo e fordismo, {\Formular{\textbf{Antonio Gramsci}}}
\item Troca de pele, {\Formular{\textbf{Tereza Yamashita}}}
\item O Surgimento da noite, {\Formular{\textbf{Pajés Parahiteri}}}
\item Contos de Sebastopol, {\Formular{\textbf{Liev Tolstói}}}
\item Um anarquista e outros contos, {\Formular{\textbf{Joseph Conrad}}}
\item Um Retrato do artista quando jovem, {\Formular{\textbf{James Joyce}}}
\item O Princípio do Estado e outros ensaios, {\Formular{\textbf{Mikhail Bakunin}}}
\item A Desmedida na medida, {\Formular{\textbf{Albert Camus}}}
\item O Chamado selvagem, {\Formular{\textbf{Jack London}}}
\item O Novo epicuro, {\Formular{\textbf{Edward Sellon}}}
\item Elogio da loucura, {\Formular{\textbf{Erasmo de Rotterdam}}}
\item Senhorita Júlia e outras peças, {\Formular{\textbf{August Strindberg}}}
\item Dublinenses, {\Formular{\textbf{James Joyce}}}
\item Don Juan ou O convidado de pedra, {\Formular{\textbf{Molière}}}
\item Manual inútil da televisão, {\Formular{\textbf{Paulo Henrique Amorim}}}
\item A Vida de Mat, {\Formular{\textbf{Mino Carta}}}
\item Baleiazzzul, {\Formular{\textbf{Sergio Zlotnic}}}
\item A Decadência do analfabetismo / A arte de birlibirloque, {\Formular{\textbf{José Bergamín}}}
\item Balada dos enforcados e outros poemas, {\Formular{\textbf{François Villon}}}
\item O Médico e o monstro, {\Formular{\textbf{Robert Louis Stevenson}}}
\item Marco Cornélio Frontão, {\Formular{\textbf{Pascal Quignard}}}
\item O Casamento do Céu e do Inferno, {\Formular{\textbf{William Blake}}}
\item O Homem da cabeça de papelão, {\Formular{\textbf{João do Rio}}}
\item Teleny, ou o reverso da medalha, {\Formular{\textbf{Oscar Wilde}}}
\item Cordel: Rodolfo Coelho Cavalcante, {\Formular{\textbf{Coelho Cavalcante}}}
\item Dicionário de História e Cultura da era Viking, {\Formular{\textbf{Johnni Langer}}}
\item Gente de Hemsö, {\Formular{\textbf{August Strindberg}}}
\item Viagem aos Estados Unidos, {\Formular{\textbf{Alexis de Tocqueville}}}
\item Sobre a utilidade e a desvantagem da história para a vida, {\Formular{\textbf{Friedrich Nietzsche}}}
\item Flossie, a Vênus de quinze anos, {\Formular{\textbf{Charles Swinburne}}}
\item Os cantos do homem-sombra
\item Escritos revolucionários, {\Formular{\textbf{Errico Malatesta}}}
\item Micromegas e outros contos, {\Formular{\textbf{Voltaire}}}
\item Descobrindo o Islã no Brasil, {\Formular{\textbf{Karla Lima}}}
\item A Cidade mágica, {\Formular{\textbf{Edith Nesbitt}}}
\item O Alienista, {\Formular{\textbf{Machado de Assis}}}
\item Cadeira de balanço, {\Formular{\textbf{Vanessa Campos Rocha; Flávio Castellan}}}
\item Inspiração nordestina, {\Formular{\textbf{Patativa do Assaré}}}
\item Coisas que a gente gosta e não gosta, {\Formular{\textbf{Laura Teixeira; Fábio Zimbres}}}
\item A Guerra começou, onde está a guerra?, {\Formular{\textbf{Albert Camus}}}
\item Poesia completa, {\Formular{\textbf{Orides Fontela}}}
\item A Volta do parafuso, {\Formular{\textbf{Henry James}}}
\item Cartas a favor da escravidão, {\Formular{\textbf{José de Alencar}}}
\item Pequeno-burgueses, {\Formular{\textbf{Maksim Górki}}}
\item Cordel : Paulo Nunes Batista, {\Formular{\textbf{Paulo Nunes Batista}}}
\item Esquimó, {\Formular{\textbf{Olivier Douzou}}}
\item Sai da frente, vaca brava!, {\Formular{\textbf{Ricardo Lísias}}}
\item Lampião... Era o cavalo do tempo atrás da besta da vida, {\Formular{\textbf{Antônio Klévisson Viana}}}
\item Cordel: Patativa do Assaré, {\Formular{\textbf{Patativa do Assaré}}}
\item Ernestine ou o nascimento do amor, {\Formular{\textbf{Stendhal}}}
\item Filadélfia, lá vou eu!, {\Formular{\textbf{Brian Friel}}}
\item Sonetos, {\Formular{\textbf{William Shakespeare}}}
\item Crônicas do crack, {\Formular{\textbf{Luis Marra}}}
\item Peixinhos, {\Formular{\textbf{Bruno Heitz}}}
\item A Última folha e outros contos, {\Formular{\textbf{O. Henry}}}
\item Contos indianos, {\Formular{\textbf{Stéphane Mallarmé}}}
\item Violência, mas para quê?, {\Formular{\textbf{Anselm Jappe}}}
\item A Vênus das peles, {\Formular{\textbf{Sacher-Leopold Von Masoch}}}
\item A Voz dos botequins e outros poemas, {\Formular{\textbf{Paul Verlaine}}}
\item Poemas, {\Formular{\textbf{Lord Byron}}}
\item A Pele do lobo e outras peças, {\Formular{\textbf{Artur Azevedo}}}
\item Explosão - Romance da etnologia, {\Formular{\textbf{Hubert Fichte}}}
\item Stephen herói, {\Formular{\textbf{James Joyce}}}
\item Diálogo imaginário entre Marx e Bakunin, {\Formular{\textbf{Maurice Cranston}}}
\item Nada ainda?, {\Formular{\textbf{Christian Voltz}}}
\item A Vênus de quinze anos (Flossie), {\Formular{\textbf{Charles Swinburne}}}
\item Os dentinhos, {\Formular{\textbf{Olivier Douzou}}}
\item Anarquismo, {\Formular{\textbf{Murray Bookchin}}}
\item Escritos sobre arte, {\Formular{\textbf{Charles Baudelaire}}}
\item Deus e o Estado, {\Formular{\textbf{Mikhail Bakunin}}}
\item Pintura e escrita do mundo flutuante, {\Formular{\textbf{Madalena Hashimoto Cordaro}}}
\item A Árvore dos cantos, {\Formular{\textbf{Pajés Parahiteri}}}
\item Poesia catalã - das origens à Guerra Civil
\item Sobre a filosofia e seu método, {\Formular{\textbf{Arthur Schopenhauer}}}
\item Pensamento político de Maquiavel, {\Formular{\textbf{Johann Fichte}}}
\item Sobre a ética, {\Formular{\textbf{Arthur Schopenhauer}}}
\item A Autobiografia do poeta-escravo, {\Formular{\textbf{Juan Francisco Manzano}}}
\item Cálcio, {\Formular{\textbf{Pádua Fernandes}}}
\item Bola de sebo e outros contos, {\Formular{\textbf{Guy de Maupassant}}}
\item Como gente grande, {\Formular{\textbf{Anouk Ricard}}}
\item O Cavalo de Ébano, {\Formular{\textbf{Richard Burton}}}
\item Nos cumes do desespero, {\Formular{\textbf{Emil Cioran}}}
\item A Vênus das peles [Bolso], {\Formular{\textbf{Leopold Von Sacher-Masoch}}}
\item Homo Pictor, {\Formular{\textbf{Christoph Wulf}}}
\item 1964
\item Desenganos da vida humana e outros poemas, {\Formular{\textbf{Gregório de Matos}}}
\item A Nostálgica e outros contos, {\Formular{\textbf{Aléxandros Papadiamántis}}}
\item Cântico dos Cânticos, {\Formular{\textbf{Salomão}}}
\item Os Sovietes traídos pelos bolcheviques, {\Formular{\textbf{Rudolf Rocker}}}
\item Autobiografia de uma pulga, {\Formular{\textbf{Stanislas de Rhodes}}}
\item Auto da barca do Inferno, {\Formular{\textbf{Gil Vicente}}}
\item A Monadologia e outros textos, {\Formular{\textbf{Gottfried Leibniz}}}
\item O Surgimento dos pássaros, {\Formular{\textbf{Pajés Parahiteri}}}
\item Contos de piratas, {\Formular{\textbf{Arthur Conan Doyle}}}
\item O Mundo ou tratado da luz, {\Formular{\textbf{René Descartes}}}
\item Manifesto comunista, {\Formular{\textbf{Karl Marx; Friedrich Engels}}}
\item Lira grega, {\Formular{\textbf{Giuliana Ragusa}}}
\item Poesia basca - das origens à Guerra Civil
\item Cordel: Klévisson Viana, {\Formular{\textbf{Klévisson Viana}}}
\item Discursos ímpios, {\Formular{\textbf{Marquês de Sade}}}
\item Cordel : Raimundo Santa Helena, {\Formular{\textbf{Raimundo Santa Helena}}}
\item Primeiro livro dos amores, {\Formular{\textbf{Ovídio}}}
\item Último reino, {\Formular{\textbf{Pascal Quignard}}}
\item Da arte de construir, {\Formular{\textbf{Leon Battista Alberti}}}
\item Frankenstein, {\Formular{\textbf{Mary Shelley}}}
\item Cordel : Zé Saldanha, {\Formular{\textbf{Zé Saldanha}}}
\item Dilma Rousseff e o ódio político, {\Formular{\textbf{Tales Ab'Sáber}}}
\item Saga dos Volsungos, {\Formular{\textbf{Anônimo}}}
\item Linear G, {\Formular{\textbf{Gilberto Mendonça Teles}}}
\item Educação e sociologia, {\Formular{\textbf{Émile Durkheim}}}
\item Histórias com dragões, príncipes e serpentes, {\Formular{\textbf{Vários}}}
\item História do boi misterioso, {\Formular{\textbf{Leandro Gomes de Barros; Irani Med}}}
\item Sobre verdade e mentira, {\Formular{\textbf{Friedrich Nietzsche}}}
\item Sermões 2, {\Formular{\textbf{Antônio Vieira}}}
\item Lisístrata, {\Formular{\textbf{Aristófanes}}}
\item Os Americanos, {\Formular{\textbf{Nathaniel Hawthorne; Edgar Allan Poe; Herman Melville}}}
\item O Sol não espera, {\Formular{\textbf{Marília Castello Branco}}}
\item O Fim do ciúme e outros contos, {\Formular{\textbf{Marcel Proust}}}
\item Álcoois, {\Formular{\textbf{Guillaume Apollinaire}}}
\item A História do planeta azul, {\Formular{\textbf{Andri Snaer Magnason}}}
\item Entre camponeses, {\Formular{\textbf{Errico Malatesta}}}
\item Ispinho e Fulô, {\Formular{\textbf{Patativa do Assaré}}}
\item Mais dia menos dia, a paixão, {\Formular{\textbf{Nelson de Oliveira}}}
\item Teogonia, {\Formular{\textbf{Hesíodo}}}
\item Ação e percepção nos processos educacionais do corpo em formação, {\Formular{\textbf{Cecília Noriko Ito Saito}}}
\item Amores e outras imagens, {\Formular{\textbf{Filóstrato}}}
\item O Fantástico reparador de feridas, {\Formular{\textbf{Brian Friel}}}
\item Mangá, {\Formular{\textbf{Sonia Bide Luyten}}}
\item Inferno, {\Formular{\textbf{August Strindberg}}}
\item Romanceiro cigano, {\Formular{\textbf{Sermões}}}
\item Sagas, {\Formular{\textbf{August Strindberg}}}
\item O Destino do erudito, {\Formular{\textbf{Johann Fichte}}}
\item Diários de Adão e Eva, {\Formular{\textbf{Mark Twain}}}
\item Habitar, {\Formular{\textbf{André Fernandes}}}
\item O Desertor, {\Formular{\textbf{Silva Alvarenga}}}
\item Os Vínculos, {\Formular{\textbf{Giordano Bruno}}}
\item O Estranho caso do Dr. Jekyll e Mr. Hyde, {\Formular{\textbf{Robert Louis Stevenson}}}
\item Sátiras, fábulas, aforismos e profecias, {\Formular{\textbf{Leonardo da Vinci}}}
\item Poesia espanhola - das origens à Guerra Civil
\item Hino a Afrodite e outros poemas, {\Formular{\textbf{Safo de Lesbos}}}
\item Revolução e liberdade, {\Formular{\textbf{Mikhail Bakunin}}}
\item Cartas do Brasil, {\Formular{\textbf{Antonio Vieira}}}
\item A Mulher que virou Tatu
\item Sermões 1, {\Formular{\textbf{Antônio Vieira}}}
\item Fé e saber, {\Formular{\textbf{G.W. Friedrich Hegel}}}
\item Negras tormentas, {\Formular{\textbf{Alexandre Samis}}}
\item Cordel: Manoel Caboclo, {\Formular{\textbf{Manoel Caboclo}}}
\item Graciliano Ramos e A Novidade, {\Formular{\textbf{Ieda Lebensztayn}}}
\item Emília Galotti, {\Formular{\textbf{Gotthold Ephraim Lessing}}}
\item Dao De Jing, {\Formular{\textbf{Lao Zi}}}
\item Histórias escondidas, {\Formular{\textbf{Odilon Moraes}}}
\item Noites egípcias e outros contos, {\Formular{\textbf{Aleksandr Púchikin}}}
\item Carmilla, a vampira de Karnstein, {\Formular{\textbf{Sheridan Le Fanu}}}
\item O desafio de Lula, {\Formular{\textbf{Mino Carta; Gianni Carta}}}
\item A Filosofia na era trágica dos gregos, {\Formular{\textbf{Friedrich Nietzsche}}}
\item O Que é bom, o que é ruim, {\Formular{\textbf{Vladimir Maiakóvski}}}
\item Em busca do Japão contemporâneo, {\Formular{\textbf{John Milton}}}
\item A Vida de H.P. Lovecraft, {\Formular{\textbf{S.T. Joshi}}}
\item A Demanda do Santo Graal, {\Formular{\textbf{Anônimo}}}
\item Trabalhos e dias, {\Formular{\textbf{Hesíodo}}}
\item Mensagem, {\Formular{\textbf{Fernando Pessoa}}}
\item Ode sobre a melancolia e outros poemas, {\Formular{\textbf{John Keats}}}
\item O Corno de si próprio e outros contos, {\Formular{\textbf{Marquês de Sade}}}
\item Hawthorne e seus musgos, {\Formular{\textbf{Herman Melville}}}
\item Memórias de um menino judeu do Bom Retiro, {\Formular{\textbf{Victor Nussenzwieg}}}
\item No coração das trevas, {\Formular{\textbf{Joseph Conrad}}}
\item Émile e Sophie ou os solitários, {\Formular{\textbf{Jean-Jaqcques Rousseau}}}
\item Investigação sobre o entendimento humano, {\Formular{\textbf{David Hume}}}
\item Ideias de canário, {\Formular{\textbf{Machado de Assis}}}
\item Eu acuso! / O processo do capitão Dreyfus, {\Formular{\textbf{Émile Zola; Rui Barbosa}}}
\item O Livro dos dragões, {\Formular{\textbf{Ovídio}}}
\item As Bacantes, {\Formular{\textbf{Eurípides}}}
\item Contos clássicos de vampiro [Bolso], {\Formular{\textbf{Lord Byron; Bram Stoker}}}
\item Sobre a liberdade, {\Formular{\textbf{Stuart Mill}}}
\item Metamorfoses, {\Formular{\textbf{Ovídio}}}
\item O Primeiro Hamlet, {\Formular{\textbf{William Shakespeare}}}
\item O Corvo, {\Formular{\textbf{Claudio Weber Abramo}}}
\item A Vida é sonho, {\Formular{\textbf{Calderón de La Barca}}}
\item Eu, {\Formular{\textbf{Augusto dos Anjos}}}
\item Cordel: Zé Vicente, {\Formular{\textbf{Zé Vicente}}}
\item Escritos sobre literatura, {\Formular{\textbf{Sigmund Freud}}}
\item Dez poemas da vizinhança vazia, {\Formular{\textbf{Iuri Pereira}}}
\item Um gato Indiscreto e outros contos, {\Formular{\textbf{Saki}}}
\item Ciclovia, {\Formular{\textbf{Ulisses Garcez}}}
\item O Livro de Monelle, {\Formular{\textbf{Marcel Schwob}}}
\item A Fábrica de robôs [Bolso], {\Formular{\textbf{Karel Tchápek}}}
\item Oração aos moços, {\Formular{\textbf{Rui Barbosa}}}
\item A Metamorfose, {\Formular{\textbf{Franz Kafka}}}
\item História de Aladim e a lâmpada maravilhosa, {\Formular{\textbf{Patativa do Assaré}}}
\item Ninfas, {\Formular{\textbf{Giorgio Agamben}}}
\item O Ladrão honesto e outros contos, {\Formular{\textbf{Fiódor Dostoiévski}}}
\item O Enigma Orides, {\Formular{\textbf{Gustavo de Castro}}}
\item A Cruzada das crianças / Vidas imaginárias, {\Formular{\textbf{Marcel Schwob}}}
\item Sobre o riso e a loucura, {\Formular{\textbf{Hipócrates}}}
\item Notas sobre o anarquismo, {\Formular{\textbf{Noam Chomsky}}}
\item Mare Nostrum, {\Formular{\textbf{Fabio Atui}}}
\item Cordel: Expedito Sebastião Da Silva, {\Formular{\textbf{Expedito Sebastião}}}
\item Mistério na zona sul, {\Formular{\textbf{Roberto Barbato Junior}}}
\item Cordel: Zé Melancia, {\Formular{\textbf{Zé Melancia}}}
\item Lulismo, carisma pop e cultura anticrítica, {\Formular{\textbf{Tales Ab'Sáber}}}
\item Perversão, {\Formular{\textbf{Robert J. Stoller}}}
\item Poesia galega - das origens à Guerra Civil
\item Naqueles morros, depois da chuva, {\Formular{\textbf{Edival Lourenço}}}
\item Os Comedores de terra, {\Formular{\textbf{Pajés Parahiteri}}}
\item Ilíada, {\Formular{\textbf{Homero}}}
\item A Semente e a torre, {\Formular{\textbf{Leonardo da Vinci}}}
\item A Farsa de Inês Pereira, {\Formular{\textbf{Gil Vicente}}}
\item Cão, {\Formular{\textbf{Rafael Mantovani}}}
\item Diário de um escritor (1873), {\Formular{\textbf{Fiódor Dostoiévski}}}
\item Carta sobre a tolerância, {\Formular{\textbf{John Locke}}}
\item Anarquia pela educação, {\Formular{\textbf{Élisée Reclus}}}
\item A Raposa sombria, {\Formular{\textbf{Sjón}}}
\item Anjos do universo, {\Formular{\textbf{Einar Már Gudmundsson}}}
\item O Indivíduo, a sociedade e o Estado e outros ensaios, {\Formular{\textbf{Emma Goldman}}}
\item A terra uma só, {\Formular{\textbf{Timóteo da Silva Verá Tupã Popyguá}}}
\item Mistério no morro do deleite, {\Formular{\textbf{Roberto Barbato Junior}}}
\item A arte de contar histórias, {\Formular{\textbf{Water Benjamin}}}
}
\end{enumerate}
\end{multicols}

\pagebreak

